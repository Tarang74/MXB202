%!TEX TS-program = xelatex
%!TEX options = -aux-directory=Debug -shell-escape -file-line-error -interaction=nonstopmode -halt-on-error -synctex=1 "%DOC%"
\documentclass{article}
\input{LaTeX-Submodule/template.tex}

% Additional packages & macros

% Header and footer
\newcommand{\unitName}{Advanced Calculus}
\newcommand{\unitTime}{Semester 2, 2023}
\newcommand{\unitCoordinator}{Dr Pascal Buenzli}
\newcommand{\documentAuthors}{Tarang Janawalkar}

\fancyhead[L]{\unitName}
\fancyhead[R]{\leftmark}
\fancyfoot[C]{\thepage}

% Copyright
\usepackage[
    type={CC},
    modifier={by-nc-sa},
    version={4.0},
    imagewidth={5em},
    hyphenation={raggedright}
]{doclicense}

\date{}

\begin{document}
%
\begin{titlepage}
    \vspace*{\fill}
    \begin{center}
        \LARGE{\textbf{\unitName}} \\[0.1in]
        \normalsize{\unitTime} \\[0.2in]
        \normalsize\textit{\unitCoordinator} \\[0.2in]
        \documentAuthors
    \end{center}
    \vspace*{\fill}
    \doclicenseThis
    \thispagestyle{empty}
\end{titlepage}
\newpage
%
\tableofcontents
\newpage
%
\section{Euclidean Space}
The Euclidean space \(\R^n\) is an \(n\)-dimensional vector space of real numbers.
This space is closed under addition and scalar multiplication.
\subsection{Operations}
\subsubsection{Addition}
The sum of two vectors \(\symbf{x}\) and \(\symbf{y}\) is defined element-wise
\begin{equation*}
    \symbf{x} + \symbf{y} = \begin{bmatrix}
        x_1 + y_1 \\
        x_2 + y_2 \\
        \vdots    \\
        x_n + y_n
    \end{bmatrix}
\end{equation*}
In a coordinate system, the vectors \(\symbf{x}\) and \(\symbf{y}\) are added tip-to-tail.
\subsubsection{Scalar Multiplication}
The scalar multiplication of a vector \(\symbf{x}\) by a scalar \(\lambda \in \R\) is defined element-wise
\begin{equation*}
    \lambda \symbf{x} = \begin{bmatrix}
        \lambda x_1 \\
        \lambda x_2 \\
        \vdots      \\
        \lambda x_n
    \end{bmatrix}
\end{equation*}
In a coordinate system, \(\lambda\) scales the vector \(\symbf{x}\) along the same line.
\subsubsection{Norm}
The norm (length) of a vector \(\symbf{x}\) is defined as
\begin{equation*}
    \norm{\symbf{x}} = \sqrt{\symbf{x} \cdot \symbf{x}} = \sqrt{\sum_{i=1}^n x_i^2}
\end{equation*}
The norm of a vector \(\symbf{x}\) is the distance from the origin to the tip of the vector.
This allows us to define the unit vector \(\hat{\symbf{x}}\) as
\begin{equation*}
    \hat{\symbf{x}} = \frac{\symbf{x}}{\norm{\symbf{x}}}
\end{equation*}
which is a vector of length 1 in the same direction as \(\symbf{x}\).
\subsubsection{Scalar Product}
The scalar product (dot product) of two vectors \(\symbf{x}\) and \(\symbf{y}\) is defined as
\begin{equation*}
    \symbf{x} \cdot \symbf{y} = \sum_{i=1}^n x_i y_i
\end{equation*}
The scalar product allows us to define the angle \(\theta\) between two vectors \(\symbf{x}\) and \(\symbf{y}\) as
\begin{equation*}
    \cos{\left( \theta \right)} = \hat{\symbf{x}} \cdot \hat{\symbf{y}}
\end{equation*}
where we use the unit vectors of \(\symbf{x}\) and \(\symbf{y}\), as the angle between two vectors is invariant under scaling.
Additionally, we can determine the projection of the vector \(\symbf{x}\) onto the vector \(\symbf{y}\) using trigonometry
\begin{equation*}
    \proj_{\symbf{y}} \left( \symbf{x} \right) = \left( \norm{\symbf{x}} \cos{\left( \theta \right)} \right) \hat{\symbf{y}} = \left( \norm{\symbf{x}} \left( \hat{\symbf{x}} \cdot \hat{\symbf{y}} \right) \right) \hat{\symbf{y}} = \left( \symbf{x} \cdot \hat{\symbf{y}} \right) \hat{\symbf{y}}
\end{equation*}
where \(\symbf{x} \cdot \hat{\symbf{y}}\) is the norm of the projection vector.
\subsection{Additional Properties}
\subsubsection{Triangle Inequality}
\begin{equation*}
    \norm{\symbf{x} + \symbf{y}} \leqslant \norm{\symbf{x}} + \norm{\symbf{y}}
\end{equation*}
\subsubsection{Inverse Triangle Inequality}
\begin{equation*}
    \norm{\symbf{x} - \symbf{y}} \geqslant \abs{\norm{\symbf{x}} - \norm{\symbf{y}}}
\end{equation*}
\subsubsection{Cauchy-Schwarz Inequality}
\begin{equation*}
    \abs{\symbf{x} \cdot \symbf{y}} \leqslant \norm{\symbf{x}} \norm{\symbf{y}}
\end{equation*}
\subsection{Multivariable Functions}
A multivariable function \(f\) maps a vector \(\symbf{x} \in \R^n\) to a real number \(f \left( \symbf{x} \right) \in \R\).
This function can be expressed in \textbf{explicit form} as
\begin{equation*}
    z = f\left( x,\: y \right)
\end{equation*}
or in \textbf{implicit form} as
\begin{equation*}
    F\left( x,\: y,\: z \right) = z - f\left( x,\: y \right)
\end{equation*}
These equations define a surface in \(\R^3\).
\subsubsection{Level Curves}
The level curves of a function \(f \left( x,\: y \right)\) are the curves in \(\R^2\) where
\begin{equation*}
    f \left( x,\: y \right) = c
\end{equation*}
where \(c\) is the height of the curve. Implicitly, this is equivalent to
\begin{equation*}
    F\left( x,\: y,\: z \right) = 0.
\end{equation*}
Level curves represent paths of equal height on the surface defined by \(z = f \left( x,\: y \right)\).
\subsection{Special Regions}
\subsubsection{Balls}
In an Euclidean space, an open ball of radius \(r > 0\) centred at a point \(\symbf{p} \in \R^n\)
is denoted \(B_r\left( \symbf{p} \right)\), and is defined as
\begin{equation*}
    B_r\left( \symbf{p} \right) = \left\{ \symbf{x} \in \R^n : \norm{\symbf{x} - \symbf{p}} < r \right\}.
\end{equation*}
This region includes all points less than a distance \(r\) from the vector \(\symbf{p}\),
where the distance is typically defined by the \(L_2\)-norm:
\begin{equation*}
    \norm{\symbf{x} - \symbf{p}}_2 = \left( \sum_{i=1}^n \left( x_i - p_i \right)^2 \right)^{1/2}.
\end{equation*}
\subsection{Mathematical Representation of Curves}
\subsubsection{Explicit Form}
A curve in \(\R^2\) can be represented in explicit form as
\begin{equation*}
    y = f\left( x \right)
\end{equation*}
but this is not possible in \(\R^3\) as a 3D curve requires two equations.
For a 2D explicit curve:
\begin{itemize}
    \item \(x\) is an independent variable such that we have 1 degree of freedom.
\end{itemize}
\subsubsection{Implicit Form}
A curve in \(\R^2\) can be represented in implicit form as
\begin{equation*}
    F\left( x,\: y \right) = 0.
\end{equation*}
In 3D, we must impose an additional equation that intersects a surface.
\begin{equation*}
    \left\{ \begin{aligned}
        F\left( x,\: y,\: z \right) & = 0 \\
        G\left( x,\: y,\: z \right) & = 0
    \end{aligned} \right.
\end{equation*}
In both cases, we have 1 degree of freedom as the degrees of freedom is the
difference between the number of variables and the number of equations.
\subsubsection{Parametric Form}
In parametric form, curves are parametrised in terms of a parameter \(t\).
In 2D, this is represented as
\begin{equation*}
    \symbf{r}\left( t \right) = \abracket*{x\left( t \right),\: y\left( t \right)}
\end{equation*}
and similarly in 3D,
\begin{equation*}
    \symbf{r}\left( t \right) = \abracket*{x\left( t \right),\: y\left( t \right),\: z\left( t \right)}
\end{equation*}
\subsection{Converting Between Representations}
\subsubsection{Explicit to Implicit}
The equation \(y = f\left( x \right)\) can always be converted to implicit form by rewriting it as
\begin{equation*}
    F\left( x,\: y \right) = y - f\left( x \right) = 0.
\end{equation*}
\subsubsection{Implicit to Explicit}
The equation \(F\left( x,\: y \right) = 0\) can be converted to explicit form if we can solve for \(y\) (or \(x\)):
\subsubsection{Parametric to Explicit/Implicit}
The equation \(\symbf{r}\left( t \right) = \abracket*{x\left( t \right),\: y\left( t \right)}\) can be written in
explicit or implicit form, if the parameter \(t\) can be eliminated from the simultaneous equations.
\subsubsection{Explicit to Parametric}
The equation \(y = f\left( x \right)\) can always be converted to parametric form by choosing the parameter \(t = x\), so that
\begin{equation*}
    \symbf{r}\left( t \right) = \abracket*{t,\: f\left( t \right)}.
\end{equation*}
\subsubsection{Implicit to Parametric}
The equation \(F\left( x,\: y \right) = 0\) can be converted to parametric form if we can find \(x = p\left( t \right)\) and \(y = q\left( t \right)\),
such that \(F\left( p\left( t \right),\: q\left( t \right) \right) = 0\), and
\begin{equation*}
    \symbf{r}\left( t \right) = \abracket*{p\left( t \right),\: q\left( t \right)}
\end{equation*}
for all \(t\).
\subsection{Paramaterisation}
To parametrise a curve, consider the following strategies:
\begin{itemize}
    \item For a closed curve, consider the polar parametrisation in terms of the angle \(\theta\):
    \begin{equation*}
        \symbf{r}\left( \theta \right) = \abracket*{R\left( \theta \right) \cos{\left( \theta \right)},\: R\left( \theta \right) \sin{\left( \theta \right)}}.
    \end{equation*}
    \item For a curve that is the intersection of two surfaces, consider one of the following mappings:
    \begin{equation*}
        x \mapsto \begin{bmatrix}
            x \\
            y\left( x \right) \\
            z\left( x \right)
        \end{bmatrix} \qquad
        y \mapsto \begin{bmatrix}
            x\left( y \right) \\
            y \\
            z\left( y \right)
        \end{bmatrix} \qquad
        z \mapsto \begin{bmatrix}
            x\left( z \right) \\
            y\left( z \right) \\
            z
        \end{bmatrix}
    \end{equation*}
    \item Otherwise, consider a vector construction.
\end{itemize}
\subsubsection{Line Segments}
To parametrise a line segment from point \(A\) to \(B\), define the parameter \(t \in \interval{0}{1}\).
Then, consider the vectors \(\symbf{a} = \overline{OA}\) and \(\symbf{b} = \overline{OB}\).
By scaling the vector from \(A\) to \(B\) by \(t\), we can parametrise the line segment as
\begin{equation*}
    \symbf{r}\left( t \right) = \symbf{a} + t \left( \symbf{b} - \symbf{a} \right) = \symbf{a} \left( 1 - t \right) + \symbf{b} t.
\end{equation*}
\subsubsection{Circles}
To parametrise a circle of radius \(R\) centred at the \(\abracket*{x_0,\: y_0}\),
first parametrise the curve in terms of the angle \(\theta\), then shift the curve by \(\abracket*{x_0,\: y_0}\).
\begin{equation*}
    \symbf{r}\left( \theta \right) =
    \begin{bmatrix}
        x_0 \\
        y_0
    \end{bmatrix} +
    \begin{bmatrix}
        R \cos{\left( \theta \right)} \\
        R \sin{\left( \theta \right)}
    \end{bmatrix}
    =
    \begin{bmatrix}
        x_0 + R \cos{\left( \theta \right)} \\
        y_0 + R \sin{\left( \theta \right)}
    \end{bmatrix}
\end{equation*}
\subsubsection{Velocity Vectors}
The velocity vector of a parametrised curve \(\symbf{r}\left( t \right) = \begin{bmatrix}
    x\left( t \right) \\
    y\left( t \right)
\end{bmatrix}\) is defined as
\begin{equation*}
    \symbf{v}\left( t \right) = \symbf{r}'\left( t \right) = \lim_{\Delta t \to 0} \frac{\symbf{r}\left( t + \Delta t \right) - \symbf{r}\left( t \right)}{\Delta t} =
    \begin{bmatrix}
        x'\left( t \right) \\
        y'\left( t \right)
    \end{bmatrix}
\end{equation*}
where \(\symbf{v}\left( t \right)\) is a tangent vector to the curve at the point \(\symbf{r}\left( t \right)\), for all \(t\).
\subsubsection{Tangent Vectors}
Following from the definition of the velocity vector, the tangent vectors of a parametrised curve
are unit vectors in the direction of the velocity vector.
\begin{equation*}
    \hat{\symbf{\tau}}\left( t \right) = \pm \frac{\symbf{v}\left( t \right)}{\norm{\symbf{v}\left( t \right)}} = \pm \hat{\symbf{v}\left( t \right)}
\end{equation*}
For a curve given in explicit form \(y = f\left( x \right)\), the tangent vectors are given by
\begin{equation*}
    \hat{\symbf{\tau}}\left( x \right) = \pm \frac{1}{\sqrt{1 + \left( f'\left( x \right) \right)^2}} \begin{bmatrix}
        1 \\
        f'\left( x \right)
    \end{bmatrix}
\end{equation*}
\end{document}
